\newcommand{\ExperienceEntry}[7]{
    % #1=employer
    % #2=position
    % #3=employment type
    % #4=duration
    % #5=start date
    % #6=end date
    % #7=location
    \abbr{
          \subsection{#1 - #2} \hfill #4 \\
          {\footnotesize #5 - #6%
            \IfStrEq{#3}{Full time}{}{ (#3)}%
            \IfStrEq{#7}{Remote}{}{ | #7}%
          }%
    }{%
        Position: #2
        Employer: #1
        Industry: Information Technologies and Services
        Employment Type: #3
        Start date: #5
        End date: #6
        Location: #7
    }%
}

\abbr{\section*{\faBriefcase \quad Experience} \label{sec:experience}}
{\# Experience}
\phantomsection \addcontentsline{toc}{section}{Experience}

\ExperienceEntry{Dotphoton}
{Full Stack Software Engineer}
{Full time}
{3 years}
{October 2022}
{Present}
{Remote}\\
{
    \raggedright
    - Lead engineer for
    \href{https://www.dotphoton.com/products/jetraw}{Jetraw}, a multi-storage
    data management platform achieving 8x image compression.\\

    - Architected and implemented full stack solutions using C++, Go,
    TypeScript, and Python.\\

    - Developed RESTful APIs, C shared libraries, and Node Addons, optimizing
    interoperability and system scalability.\\

    - Designed cloud-native CI/CD workflows using GitHub Actions and Jenkins to
    automate robust deployment pipelines.\\
}
\medskip

\ExperienceEntry{ProntoPro}
{Full Stack Software Engineer in Marketing Automation}
{Part time}
{7 months}
{October 2020}
{May 2021}
{Remote}\\
{
    \raggedright
    - Integrated multiple marketing automation tools with an in-house MySQL database using Python, pandas, and SQLAlchemy.
	- Implemented automated testing with pytest and GitLab CI/CD.
	- Managed projects and led a team of junior engineers.
}
\medskip

\ExperienceEntry{ProntoPro}
{Junior Full Stack Software Engineer in Marketing Automation} 
{Full time}
{11 months}
{November 2019}
{October 2020}
{Milan, Italy} \\
{
    \raggedright
	- Implemented data acquisition pipelines with web scraping tools into an
	in-house MySQL database. \\
    - Delivered a Full Stack web application with Django and React used as an
    internal marketing tool.
}
\medskip

\ExperienceEntry{Stadler Rail Valencia}
{Junior Software Engineer}
{Part time}
{6 months}
{September 2018}
{Mar 2019}
{Valencia, Spain} \\
{
    \raggedright
    - Developed predictive maintenance software with MATLAB, including GUI design.
}
\medskip